% !TeX spellcheck = en_US
\documentclass[]{article}

\usepackage[natbib]{TiPi}

\title{Scaling to integer}
\author{Éric T.}


\newcommand*{\Lag}{\mathscr{L}}
\newcommand*{\IntRange}[1]{[\![#1]\!]}
\newcommand*{\half}{\ensuremath{{}^1\!/\!{}_2}}
\renewcommand*{\proxy}[1]{\widetilde{#1}}

\begin{document}

\maketitle

\begin{abstract}
We discuss the best way to rescale floating-point data to integer values by an
affine transform.
\end{abstract}

\section{Digitization}

Assume that we have data values $\V d = \{d_1,d_2,\ldots,d_m\} \in \Reals^m$
which we want to approximate as integers $\V k = \{k_1,k_2,\ldots,k_m\} \in
\IntRange{k_\Tag{min},k_\Tag{max}}^m\subset\Integers^m$ through the affine transform:
\begin{equation}
  \label{eq:approx-data}
  \V d \approx \proxy{\V d} = \alpha\,\V k + \beta\,\One \, .
\end{equation}
If $\alpha \not= 0$, a possible approximate reciprocal formula to compute $\V
k$ is given by:
\begin{equation}
  \label{eq:digitization}
  \V k = \Round{(\V d - \beta\,\One)/\alpha} \, ,
\end{equation}
where $\Round{\cdot}: \Reals\mapsto\Integers$ is applied elementwise and rounds
its argument(s) to the nearest integer(s).  This function has the following
properties:
\begin{equation}
  \label{eq:rounding}
  u - 1/2 < \Round{u} \le u + 1/2
  \quad\Longleftrightarrow \quad
  \Round{u} - 1/2 \le u < \Round{u} + 1/2 \, ,
\end{equation}
for any $u\in\Reals$.


\section{Criteria for the affine parameters}

A first suggestion is to find the parameters $\alpha$ and $\beta$ such that the
approximation in Eq.~(\ref{eq:approx-data}) yields the least worst error:
\begin{equation}
  \label{eq:criterion}
  (\estim{\alpha},\estim{\beta})
  = \argmin_{\alpha,\beta} \Norm[\big]{\V d - \proxy{\V d}}_{\infty} \, .
\end{equation}
Note that we could have also required to optimize the criterion with respect to
the values of $\V k$ but the problem would have been far more complex.  The
assumed digitization formula~(\ref{eq:digitization}) to compute $\V k$ given
$\V d$, $\alpha$ and $\beta$ is only a possibility but it is simple and
uniformly accurate.

Another point to consider is the stability of the transform when
Eq.~(\ref{eq:approx-data}) and Eq.~(\ref{eq:digitization}) are alternately
applied several times with $\alpha$ and $\beta$ computed each time according to
the actual data range.  To avoid any drift, it may be advisable that the choice
of $\alpha$ and $\beta$ ensures that some determined data value remains exactly
represented.  This suggests to choose $\beta$ to be a multiple of $\alpha$ as
this implies that, for instance, a value of zero for the data will always be
exactly represented.  \oops{Study the successive transforms and check that no
shrink occurs.} 

To simplify the reasoning we assume in the sequel that $\alpha \ge 0$ and that
the data bounds:
\begin{align}
   d_\Tag{min} &\bydef \min_{i\in\IntRange{1,m}} d_i \, , \\ 
   d_\Tag{max} &\bydef \max_{i\in\IntRange{1,m}} d_i \, , 
\end{align}
are both finite.  We also note that if $d_\Tag{max} = d_\Tag{min}$, we can
choose $\estim{\alpha}$, $\estim{\beta}$ and $\V k = \Zero$ such that the worst
error is exactly zero\footnote{ For instance, if $0 \in
\IntRange{k_\Tag{min},k_\Tag{max}}$, taking $\estim{\alpha} = 1$,
$\estim{\beta} = d_\Tag{max} = d_\Tag{min}$ and $\V k = \Zero$ yields
$\Norm[\big]{\V d - \proxy{\V d}}_{\infty} = 0$; otherwise, if $0 \not\in
\IntRange{k_\Tag{min},k_\Tag{max}}$, taking $\estim{\alpha} = 0$,
$\estim{\beta} = d_\Tag{max} = d_\Tag{min}$ and any $\V k \in
\IntRange{k_\Tag{min},k_\Tag{max}}^m$ also yields $\Norm[\big]{\V d - \proxy{\V
d}}_{\infty} = 0$.}.  In what follows, we therefore consider that $d_\Tag{max}
> d_\Tag{min}$.


\section{Choosing the scale}

If $\alpha=0$ then $\proxy{\V d}$ given by Eq.~(\ref{eq:approx-data}) does not
depend on $\V k$ and the worst error is:
\begin{displaymath}
  \Norm[\big]{\V d - \proxy{\V d}}_{\infty} = \max\Brace{
    \Abs{d_\Tag{min} - \beta}, \Abs{d_\Tag{max} - \beta}
  } \qquad\text{(when $\alpha = 0$)} \, .
\end{displaymath}
This error is obviously minimized when $\beta$ is the central value of the data range:
\begin{displaymath}
  \estim{\beta} = (d_\Tag{min} + d_\Tag{max})/2
  \qquad\text{(when $\estim{\alpha} = 0$)} \, .
\end{displaymath}

We now consider the other possibility and assume that $\alpha > 0$.  For $\V k$
given by Eq.~(\ref{eq:digitization}), the pointwise error is then given by:
\begin{displaymath}
  \V d - \proxy{\V d} = \alpha\,\Paren[\big]{\V u - \Round{\V u}} \, ,
\end{displaymath}
with $\V u \bydef (\V d - \beta\,\One)/\alpha$.  Since, from
Eq.~(\ref{eq:rounding}), $-1/2 \le u - \Round{u} < 1/2$, the worst error is
then:
\begin{equation}
  \label{eq:worst-error}
  \Norm[\big]{\V d - \proxy{\V d}}_{\infty} = \Abs{\alpha}/2 = \alpha/2 \, ,
\end{equation}
where the last equality follows from our assumption that $\alpha \ge 0$.  As a
result of the chosen digitization formula~(\ref{eq:digitization}), the worst
error does not depend on $\beta$ but solely on $\alpha$.  According to
Eq.~(\ref{eq:worst-error}), to have the least worst error, we should choose the
smallest $\alpha$ (in magnitude) such that all $\V k$ given by
Eq.~(\ref{eq:digitization}) are in the range
$\IntRange{k_\Tag{min},k_\Tag{max}}$.  This corresponds to the intuition that
the smaller the discretization step $\alpha$, the more accurate the resulting
digitization.

Whatever $\beta$, the smaller the magnitude of $\alpha$ the larger is the
interval spanned by the values of $\V k$ computed according to
Eq.~(\ref{eq:digitization}), thus the smallest possible $\alpha$ is such that
the bounds $k_\Tag{min}$ and $k_\Tag{max}$ are reached when the data span the
range $[d_\Tag{min},d_\Tag{max}]$.  Since the mapping $\V d \mapsto \V k$
implemented by the digitization formula~(\ref{eq:digitization}) is
monotonically increasing (because $\alpha > 0$), the following constraints must
hold:
\begin{align}
   &\left\{
   \begin{array}{l}
   \Round{(d_\Tag{min} - \beta)/\alpha} = k_\Tag{min} \\[1ex]
   \Round{(d_\Tag{max} - \beta)/\alpha} = k_\Tag{max}
   \end{array}\right. \notag  \\
   \Longleftrightarrow \quad
   &\left\{
   \begin{array}{l}
   k_\Tag{min} - 1/2 \le (d_\Tag{min} - \beta)/\alpha < k_\Tag{min} + 1/2 \\[1ex]
   k_\Tag{max} - 1/2 \le (d_\Tag{max} - \beta)/\alpha < k_\Tag{max} + 1/2
   \end{array}\right.
   \label{eq:bounds}
\end{align}
taking the difference between the two last pairs of inequalities yields:
\begin{displaymath}
  k_\Tag{max} - k_\Tag{min} - 1 < (d_\Tag{max} - d_\Tag{min})/\alpha <
  k_\Tag{max} - k_\Tag{min} + 1 \, .
\end{displaymath}
As we assumed that $d_\Tag{max} > d_\Tag{min}$, the above inequalities become:
\begin{displaymath}
  \frac{d_\Tag{max} - d_\Tag{min}}{k_\Tag{max} - k_\Tag{min} + 1}
  < \alpha <
  \frac{d_\Tag{max} - d_\Tag{min}}{k_\Tag{max} - k_\Tag{min} - 1} \, .
\end{displaymath}
Thus:
\begin{equation}
  \label{eq:best-alpha}
  \boxed{
    \alpha = \frac{d_\Tag{max} - d_\Tag{min}}
    {k_\Tag{max} - k_\Tag{min} + \eta}
  } \, ,
\end{equation}
with $\eta \in (-1,1)$.  The smallest possible $\alpha$ corresponds to $\eta =
1$ but this is a strict lower bound.  For now, we keep the freedom to choose
$\eta \in (-1,1)$ and consider how to determine the bias $\beta$.


\section{Choosing the bias}

The inequalities in Eq.~(\ref{eq:bounds}) can be combined to bound the value of
$\beta/\alpha$:
\begin{displaymath}
  \max\Brace{\sigma_0, \sigma_1} - 1/2
  < \beta/\alpha \le
  \min\Brace{\sigma_0, \sigma_1} + 1/2 \, ,
\end{displaymath}
with $\sigma_0 = d_\Tag{min}/\alpha - k_\Tag{min}$ and $\sigma_1 =
d_\Tag{max}/\alpha - k_\Tag{max}$.  The difference $\sigma_1 - \sigma_0$ has a
simple expression for $\alpha$ given by Eq.~(\ref{eq:best-alpha}):
\begin{displaymath}
  \sigma_1 - \sigma_0 =
  (d_\Tag{max} - d_\Tag{min})/\alpha - k_\Tag{max} + k_\Tag{min} = \eta \, ,
\end{displaymath}
and putting all together yields:
\begin{equation}
  \boxed{
    \gamma_0 < \beta/\alpha \le \gamma_1
  } \, ,
  \label{eq:beta-bounds}
\end{equation}
with:
\begin{align}
  \gamma_0 &= \max\Brace{\sigma_0, \sigma_1} - 1/2 \notag \\
  %&= \sigma_0 + \max\Brace{0, \eta} - 1/2 \notag \\
  &= \sigma_0 + (\eta)_{+} - 1/2
   = \sigma_1 - (\eta)_{-} - 1/2 \, , \\
  \gamma_1 &= \min\Brace{\sigma_0, \sigma_1} + 1/2 \notag \\
  %&= \sigma_0 + \min\Brace{0, \eta} + 1/2 \notag \\
  &= \sigma_0 + (\eta)_{-} + 1/2
   = \sigma_1 - (\eta)_{+} + 1/2 \, ,
\end{align}
and where $(\eta)_{+} = \max\Brace{0, \eta}$ and $(\eta)_{-} = \min\Brace{0,
\eta}$.  The range of possible values for $\gamma = \beta/\alpha$ is the
semi-open interval $(\gamma_0,\gamma_1]$.  The width of the interval is:
\begin{displaymath}
  \gamma_1 - \gamma_0 = 1 - (\eta)_{+} + (\eta)_{-} = 1 - \Abs{\eta} \, .
\end{displaymath}
Since $\eta \in (-1,1)$, $\gamma_1 - \gamma_0  \in (0,1]$ so the semi-open
interval $(\gamma_0,\gamma_1]$ is always non-empty.  The center of the interval
is:
\begin{displaymath}
  \gamma_\Tag{cen} = (\gamma_0 + \gamma_1)/2 = \sigma_0 + \eta/2 = \sigma_1 - \eta/2 \, .
\end{displaymath}

If $d_\Tag{max} > d_\Tag{min}$ and $k_\Tag{max} > k_\Tag{min}$, we have the
flexibility to choose $\eta$ in the range $(-1,1)$.   We can either:
\begin{itemize}
\item Take $\eta = 0$, then $\gamma_1 - \gamma_0 = 1$ and there is thus exactly
one integer value in the range  $(\gamma_0,\gamma_1]$ which is
$\Floor{\gamma_1}$.  This let us choose $\gamma=\beta/\alpha$ to be integer so
that a zero in the data is exactly represented in the digitized data.  To have
$\gamma$ integer, we can take $\gamma = \Floor{\gamma_1}$ but noting that, when
$\eta=0$, we have:
\begin{equation}
	\sigma_0 = \sigma_1 = \gamma_\Tag{cen}
	= \frac{d_\Tag{min}\,k_\Tag{max} - d_\Tag{max}\,k_\Tag{min}}
	       {d_\Tag{max} - d_\Tag{min}} \, ,
	       \label{eq:gamma-center}
\end{equation}
which belongs to $(\gamma_0,\gamma_1]$ and thus $\Round{\gamma_\Tag{cen}} =
\Floor{\gamma_1}$.  A possibility is thus to take $\eta=0$, and $\gamma =
\Round{\gamma_\Tag{cen}}$ or $\gamma = \gamma_\Tag{cen}$ depending whether or
not we want to exactly represent specific data values such as zero.

\item Choose $\eta \rightarrow 1^-$ to minimize the worst error which is
slightly better than the previous case but imposes $\gamma = \beta/\alpha
\rightarrow \gamma_1^-$ (or equivalently $\gamma = \beta/\alpha \rightarrow
\gamma_0^+$) which is not guaranteed to be integer.  In practice, the value of
$\eta$ must be such that the denominator in Eq.~(\ref{eq:best-alpha}) is as
close as possible but numerically strictly smaller than $k_\Tag{max} -
k_\Tag{min} + 1$, this leads to take $\eta$ such that:
\begin{align}
  &k_\Tag{max} - k_\Tag{min} + \eta
  = (k_\Tag{max} - k_\Tag{min} + 1)\,(1 - \varepsilon) \notag \\
  \Longrightarrow\quad& \eta = 1 - (k_\Tag{max} - k_\Tag{min} + 1)\,\varepsilon \, ,
  \label{eq:max-eta}
\end{align}
with $\varepsilon$ the smallest value such that $1\pm\varepsilon$ is
numerically different from 1.  The value of $\eta$ in Eq.~(\ref{eq:max-eta}) is
nonnegative if the number of digitization levels $k_\Tag{max} - k_\Tag{min} +
1$ at most $1/\varepsilon$.  For larger number of digitization levels taking
$\eta=0$ is better.  For 64-bit IEEE floating point values, $\varepsilon =
2^{-52}$ and thus using Eq.~(\ref{eq:max-eta}) imposes to digitize to at most
52-bit integers.

\item Taking $\eta < 0$ is worst than the previous cases so must be
disregarded.
\end{itemize}

To summarize, taking $\eta = 0$ is only slightly better than $\eta->1^-$ in
terms of least worst error but is more general (it applies to any number of
digitization levels) and offers more flexibility for choosing the bias $\beta$,
for instance we can have $\beta$ a multiple of $\alpha$.


\section{Stable transform}

As the computations only depend on the four given parameters, $k_\Tag{min}$,
$k_\Tag{max}$, $d_\Tag{min}$ and $d_\Tag{max}$, a simple constraint to preserve
the stability of the resulting transforms can be guaranteed if the data bounds
are exactly preserved, \emph{i.e.}:
\begin{displaymath}
  \left\{
  \begin{array}{l}
    d_\Tag{min} = \alpha\,k_0 + \beta \\[1ex]
    d_\Tag{max} = \alpha\,k_1 + \beta
  \end{array}\right.
  \quad\Longleftrightarrow \quad
  \left\{
  \begin{array}{l}
    \displaystyle
    \alpha = \frac{d_\Tag{max} - d_\Tag{min}}{k_1 - k_0} \\[2ex]
    \displaystyle
    \beta = \frac{d_\Tag{min}\,k_1 - d_\Tag{max}\,k_0}{k_1 - k_0}
  \end{array}
  \right.
\end{displaymath}
for some $(k_0,k_1) \in \IntRange{k_\Tag{min},k_\Tag{max}}^2$ and assuming that
$k_0\not=k_1$.  As discussed above the smaller $\alpha$ the more accurate the
digitization, thus $k_0 = k_\Tag{min}$ and $k_1 = k_\Tag{max}$ is the best
choice if we further impose that $k_0 < k_1$.   In that case the digitization
parameters write:
\begin{equation}
  \left\{
  \begin{array}{l}
    \displaystyle
    \alpha = \frac{d_\Tag{max} - d_\Tag{min}}
                  {k_\Tag{max} - k_\Tag{min}} \\[2ex]
    \displaystyle
    \beta = \frac{d_\Tag{min}\,k_\Tag{max} - d_\Tag{max}\,k_\Tag{min}}
                 {k_\Tag{max} - k_\Tag{min}} \\[2ex]
    \displaystyle
    \gamma = \beta/\alpha
    = \frac{d_\Tag{min}\,k_\Tag{max} - d_\Tag{max}\,k_\Tag{min}}
           {d_\Tag{max} - d_\Tag{min}}
  \end{array}
  \right.
  \label{eq:exact-interpolation}
\end{equation}
This yields the same value for $\alpha$ as in Eq.~(\ref{eq:best-alpha}) with
$\eta = 0$ and $\gamma = \beta/\alpha = \gamma_\Tag{cen}$ with
$\gamma_\Tag{cen}$ given in Eq.~(\ref{eq:gamma-center}).  In other words, the
parameters derived from exact interpolation of the bounds are compatible with
Equations~(\ref{eq:best-alpha}) and (\ref{eq:beta-bounds}) for $\eta = 0$.


\section{Proposed parameters}

To summarize all previous considerations, taking $\eta = 0$ seems the best
choice. Assuming $k_\Tag{max} > k_\Tag{min}$ and $d_\Tag{max} > d_\Tag{min}$, a
good selection of digitization parameters is then:
\begin{equation}
  \boxed{
  \begin{array}{l}
    \displaystyle
    \alpha = \frac{d_\Tag{max} - d_\Tag{min}}
                  {k_\Tag{max} - k_\Tag{min}} \, , \\[2ex]
    \displaystyle
    \gamma_\Tag{cen} = \frac{d_\Tag{min}\,k_\Tag{max} - d_\Tag{max}\,k_\Tag{min}}
                            {d_\Tag{max} - d_\Tag{min}} \, , \\[2ex]
    \displaystyle
    \beta = \alpha\times\begin{cases}
    \Round{\gamma_\Tag{cen}} & \text{to preserve zeros,} \\
    \gamma_\Tag{cen} & \text{to preserve data bounds.} \\
    \end{cases}
  \end{array}
  }
\end{equation}
When $k_\Tag{max} > k_\Tag{min}$ and $d_\Tag{max} = d_\Tag{min}$, taking
$\alpha = 0$ and $\beta = d_\Tag{max} = d_\Tag{min}$ yields an exact
representation.  Finally, when $k_\Tag{max} = k_\Tag{min}$ (which means that
the values are digitized on a single level!), $\alpha = 0$ and $\beta =
(d_\Tag{max} + d_\Tag{min})/2$ yields the least worst error.


\section{Digitization with non-finite values}

In order to cope with non-finite data (NaN or $\pm\infty$), the following
digitization rule can be applied:
\begin{equation}
  \label{eq:complex-digitization}
  k_i = \begin{cases}
     k_\Tag{nan}     & \text{if $d_i$ is not a number;}\\
     k_\Tag{+\infty} & \text{if $d_i > d_\Tag{max}$;}\\
     k_\Tag{-\infty} & \text{if $d_i < d_\Tag{min}$;}\\
     \Round{(d_i - \beta)/\alpha} & \text{otherwise.}
  \end{cases}
\end{equation}
where $k_\Tag{nan}$, $k_\Tag{+\infty}$ and $k_\Tag{-\infty}$ are chosen
integers while $d_\Tag{min}$ and $d_\Tag{max}$ (with $d_\Tag{max} \ge
d_\Tag{min}$) are both finite and specify the range of valid data values. 
\oops{Perhaps add cases for $k_\Tag{min}$ and $k_\Tag{max}$ to avoid unwanted
behavior due to rounding errors.}

The integers $k_\Tag{nan}$, $k_\Tag{+\infty}$ and $k_\Tag{-\infty}$ can be
chosen outside the range $\IntRange{k_\Tag{min},k_\Tag{max}}$ (thus reserving a
few values of the integer range for representing theses special cases) but this
is not mandatory.  For instance, if we take $k_\Tag{-\infty} = k_\Tag{min}$ and
$k_\Tag{+\infty} = k_\Tag{max}$, then the above rule implements \emph{data
clipping}:
\begin{equation}
  \label{eq:digitization-with-clipping}
  k_i = \begin{cases}
     k_\Tag{nan} & \text{if $d_i$ is not a number;}\\
     k_\Tag{min} & \text{if $d_i \le d_\Tag{min}$;}\\
     k_\Tag{max} & \text{if $d_i \ge d_\Tag{max}$;}\\
     \Round{(d_i - \beta)/\alpha} & \text{otherwise.}
  \end{cases}
\end{equation}

\bibliographystyle{plainnat}
\bibliography{journals-short,biblio}

\end{document}
